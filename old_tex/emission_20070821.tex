\documentclass[11pt]{article}

\input{format}

\begin{document}

\title{A Guide to Emission and Absorption Lines and ``What they mean''.}
\author{Nicholas P. Ross}
\date{\today}
\maketitle


\begin{abstract}
This is a simple document which will hopefully\/eventually be a pretty complete
list of various emission lines and `what they mean'. That is to say, when a 
paper reports a flux of a certain line, why is that line special? Is it because
that line indicates current star formation, past star formation or maybe
AGN activity. We shall hopefully discuss a few line ratios and spectral
diagnostic plots. 
\end{abstract}

%Section heading
\section{The Lines}

\noindent
N.B. 13.6 eV $\equiv$

\begin{landscape}
\begin{table}
  \caption{The Lines}
  \label{tab:the_lines}
  \begin{center}
    \begin{tabular}{llllll} 
      \hline
      \hline
      Name & Wavelength / \AA & Transition & Rest Passband & 
      Interpreation & Reference \\
      \hline
      Lyman-$\alpha$ & 1215.67 & 2 to 1        & $\sim$FUV & Major QSO line       & 1 \\
      Lyman-$\beta$  & 1215.67 & 3 to 1        & $\sim$FUV &        & 1 \\
      Lyman-$\gamma$ & 1215.67 & 4 to 1        & $\sim$FUV &        & 1 \\
      Lyman Limit    & 1215.67 & $\infty$ to 1 & $\sim$FUV &        & 1 \\
      \hline
      H-$\alpha$     & 6563.   & 3 to 2        & R,r       & Recent major SF or AGN activity & 2 \\
      H-$\beta$      & 4861.   & 4 to 2        & B,V,g     &  & 2 \\
      H-$\gamma$     & 4341.   & 5 to 2        & U,B,u     &  & 2 \\
      H-$\delta$     & 4102.   & 6 to 2        & $\sim$FUV & Previous SF history  & 3 \\
      Balmer Limit   & 3646.   & $\infty$ to 2 & $\sim$FUV &  & 2 \\
      \hline
      HI              & 3646.   & $\infty$ to 2 & $\sim$FUV &  & 2 \\
      HII             & 3646.   & $\infty$ to 2 & $\sim$FUV &  & 2 \\
      \hline
      HeI              & 3646.   & $\infty$ to 2 & $\sim$FUV &  & 2 \\
      HeII             & 3646.   & $\infty$ to 2 & $\sim$FUV &  & 2 \\
      HeIII            & 3646.   & $\infty$ to 2 & $\sim$FUV &  & 2 \\
      \hline
      CIV              & 3646.   & $\infty$ to 2 & $\sim$FUV & Major QSO line & 2 \\
      \hline
      OII              & 3646.   & $\infty$ to 2 & $\sim$FUV & Major QSO line & 2 \\
      \hline
      OIII             & 3646.   & $\infty$ to 2 & $\sim$FUV & Recent major SF line & 2 \\ 
      OIII             & 5007.   & $\infty$ to 2 & $\sim$FUV & Recent major SF line & 2 \\
      \hline
      Ca H             & 3999.   & $\infty$ to 2 & $\sim$FUV & Old stellar pop & 3 \\
      Ca K             & 4001.   & $\infty$ to 2 & $\sim$FUV & Old stellar pop & 3 \\
      \hline
      NII              & 5007.   & $\infty$ to 2 & $\sim$FUV &  & 2 \\
      \hline
      NeV              & 3646.   & $\infty$ to 2 & $\sim$FUV & Major QSO line & 2 \\
      \hline
      [OIII $\lambda$ 5007/ H$\beta$]   &    & $\infty$ to 2 & $\sim$FUV & Major QSO line & 2 \\
      \hline
      [NII $\lambda$ 6583/ H$\alpha$]   &    & $\infty$ to 2 & $\sim$FUV & Major QSO line & 2 \\
      \hline
      [$\alpha$/Fe]             & 3646.   & $\infty$ to 2 & $\sim$FUV & Major QSO line & 2 \\
      \hline
      \hline
% GALEX FUV: 1350-1750, % GALEX NUV: 1750-2800
%Lyman alpha forest is the sum of absorption lines arising from the Lyman alpha transition of the neutral hydrogen in the spectra of distant galaxies and quasars.
    \end{tabular}
  \end{center}
\end{table}
\end{landscape}

\noindent
E$+$A (e$+$a) galaxies have... (Roseboom et al. 2006 and refs therein).\\
k$+$a         galaxies have... (Roseboom et al. 2006 and refs therein).\\



\section{References}

\begin{table}
  \caption{The Refs}
  \label{tab:the_ref}
  \begin{center}
    \begin{tabular}{llllll}
      \hline
      \hline 
      Name             & Year & Journal & Volume & Page & Section(s) \\
      \hline
      Croom et al.     & 2004 & MNRAS      & 375        & 600  & ???? \\
      Kriek et al.     & 2007 & astro-ph   & 0611724    & v4   & 3 \\
      Roseboom et al.  & 2006 & MNRAS      &            &      &   \\
      \hline
      \hline
    \end{tabular}
  \end{center}
\end{table}


\end{document}

