\documentclass[11pt]{article}
\setlength {\textwidth}{180mm} 
\setlength {\textheight}{260mm}
\topmargin=-35.00mm
\oddsidemargin=-10.00mm
\pagestyle{empty}

\input{format}

\begin{document}

\title{A Guide to Emission and Absorption Lines and ``What they mean''.}
\author{Nicholas P. Ross}
\date{\today}
\maketitle


\begin{abstract}
This is a simple document which will hopefully\/eventually be a pretty complete
list of various emission lines and `what they mean'. That is to say, when a 
paper reports a flux of a certain line, why is that line special? Is it because
that line indicates current star formation, past star formation or maybe
AGN activity. We shall hopefully discuss a few line ratios and spectral
diagnostic plots. 
\end{abstract}

%Section heading
\section{The Lines}

\noindent
N.B. 13.6 eV $\equiv$

\begin{landscape}
\begin{table}
  \caption{The Lines}
  \label{tab:the_lines}
  \begin{center}
    \begin{tabular}{lrllll} 
      \hline
      \hline
      Name & Wavelength / \AA & Transition & Rest Passband & 
      Interpreation & Reference \\
      \hline
      Lyman-$\alpha$ & 1215.67 & 2 to 1        & $\sim$FUV & Major QSO line       & 1 \\
      Lyman-$\beta$  & 1025.18 & 3 to 1        & $\sim$FUV &        & 1 \\
      Lyman-$\gamma$ &  972.02 & 4 to 1        & $\sim$FUV &        & 1 \\
      Lyman Limit    &  911.27 & $\infty$ to 1 & $\sim$FUV &        & 1 \\
      \hline
      H-$\alpha$     & 6563.   & 3 to 2        & R,r       & Recent major SF or AGN activity & 2 \\
      H-$\beta$      & 4861.   & 4 to 2        & B,V,g     &  & 2 \\
      H-$\gamma$     & 4341.   & 5 to 2        & U,B,u     &  & 2 \\
      H-$\delta$     & 4102.   & 6 to 2        & $\sim$FUV & Previous SF history  & 3 \\
      Balmer Limit   & 3646.   & $\infty$ to 2 & $\sim$FUV &  & 2 \\
      \hline
      HI              & 3646.   & $\infty$ to 2 & $\sim$FUV &  & 2 \\
      HII             & 3646.   & $\infty$ to 2 & $\sim$FUV &  & 2 \\
      \hline
      HeI              & 3646.   & $\infty$ to 2 & $\sim$FUV &  & 2 \\
      HeII             & 3646.   & $\infty$ to 2 & $\sim$FUV &  & 2 \\
      HeIII            & 3646.   & $\infty$ to 2 & $\sim$FUV &  & 2 \\
      \hline
      CIV              & 3646.   & $\infty$ to 2 & $\sim$FUV & Major QSO line & 2 \\
      \hline
      OII              & 3646.   & $\infty$ to 2 & $\sim$FUV & Major QSO line & 2 \\
      \hline
      OIII             & 3646.   & $\infty$ to 2 & $\sim$FUV & Recent major SF line & 2 \\ 
      OIII             & 5007.   & $\infty$ to 2 & $\sim$FUV & Recent major SF line & 2 \\
      \hline
      Ca II H          & 3999.   & $\infty$ to 2 & $\sim$FUV & Old stellar pop & 3 \\
      Ca II K          & 4001.   & $\infty$ to 2 & $\sim$FUV & Old stellar pop & 3 \\
      \hline
      NII              & 5007.   & $\infty$ to 2 & $\sim$FUV &  & 2 \\
      \hline
      NeV              & 3646.   & $\infty$ to 2 & $\sim$FUV & Major QSO line & 2 \\
      \hline
      $[$OIII $\lambda$ 5007/ H$\beta]$ &   &  &   & ``BPT'' diagram reliable tool for determining source & 2, 4, 5 \\
      $[$NII $\lambda$ 6583/ H$\alpha]$ & & & & of line emission from a galaxy visually differentiate & 2,4,5 \\
                                      & & & & between Seyferts, LINERs and SF gals. However, only at & \\
                                      & & & & ``low'' redshifts since need H$\alpha$, (not at $z\sim1$). & \\
                                      & & & &  Modified BPT with $(U-B)$ colour replacing & \\
                                      & & & & $[$NII $\lambda$ 6583/ H$\alpha]$ e.g. Montero-Dorta, 0801.2769. & \\
      \hline
      [SII $\lambda$ 6583/ H$\alpha$]   &    & $\infty$ to 2 & $\sim$FUV & Major QSO line & 2,4. 5  \\
      \hline
      [$\alpha$/Fe]             & 3646.   & $\infty$ to 2 & $\sim$FUV & Major QSO line & 2 \\
      \hline
      NV               & 1???.67 & 2 to 1        & $\sim$FUV & Major QSO line       & 1 \\
      SiIV             & 1???.67 & 2 to 1        & $\sim$FUV & Major QSO line       & 1 \\
      CIV              & 1???.67 & 2 to 1        & $\sim$FUV & Major QSO line       & 1 \\
      CIII]            & 1???.67 & 2 to 1        & $\sim$FUV & Major QSO line       & 1 \\
      MgII             & 1???.67 & 2 to 1        & $\sim$FUV & Major QSO line       & 1 \\
      \hline
      \hline
% GALEX FUV: 1350-1750, % GALEX NUV: 1750-2800
%Lyman alpha forest is the sum of absorption lines arising from the Lyman alpha transition of the neutral hydrogen in the spectra of distant galaxies and quasars.
    \end{tabular}
  \end{center}
\end{table}
\end{landscape}


\begin{table}
  \caption{The Lines, in increasing Wavelength (Basis for this table from 
  SDSS SkyServer Schema Browser, SpecLineNames view {\tt http://casjobs.sdss.org/dr6/en/help/browser/browser.asp}) }
  \label{tab:the_lines}
  \begin{center}
    \begin{tabular}{lll} 
      \hline
      \hline
name &	value &	description \\
      \hline
UNKNOWN	   &    0    & 	0.00 \\
OVI\_1033   &	1033 &	1033.82 \\
Lya\_1215   &	1215 &	1215.67 \\
NV\_1241    &   1241 &	1240.81 \\
OI\_1306    &   1306 &	1305.53 \\
CII\_1335   &	1335 &	1335.31 \\
SiIV\_1398  &	1398 &	1397.61 \\
SiIV\_OIV\_1400 & 1400 &  1399.80 \\
CIV\_1549   &   1549 &	1549.48 \\
HeII\_1640  &	1640 &	1640.40 \\
OIII\_1666  &	1666 &	1665.85 \\
AlIII\_1857 &	1857 &	1857.40 \\
CIII\_1909  &	1909 &	1908.73 \\
CII\_2326   &	2326 &	2326.00 \\
NeIV\_2439  &	2439 &	2439.50 \\
MgII\_2799  &	2799 &	2799.12 \\
NeV\_3347   &	3347 &	3346.79 \\
NeV\_3427   &	3427 &	3426.85 \\
OII\_3727   &	3727 &  3727.09 \\
OII\_3730   &	3730 &	3729.88 \\
Hh\_3799    &   3799 &  3798.98 \\
Oy\_3836    &   3836 &	3836.47 \\
HeI\_3889   &	3889 &	3889.00 \\
CaII K\_3935 &  3935 &	3934.78 \\
CAII H\_3970 &  3970 &	3969.59 \\
He\_3971    &   3971 &	3971.19 \\
SII\_4072   &	4072 &	4072.30 \\
Hd\_4103    &   4103 &	4102.89 \\
G\_4306	    &   4306 &	4305.61 \\
Hg\_4342    &   4342 &	4341.68 \\
OIII\_4364  &	4364 &	4364.44 \\
Hb\_4863    &   4863 &  4862.68 \\
OIII\_4933  &	4933 &	4932.60 \\
OIII\_4960  &	4960 &  4960.30 \\
OIII\_5008  &	5008 &  5008.24 \\

     \hline
      \hline
 \end{tabular}
   \end{center}
\end{table}


\begin{table}
  \caption{The Lines, in increasing Wavelength (Basis for this table from 
  SDSS SkyServer Schema Browser, SpecLineNames view {\tt http://casjobs.sdss.org/dr6/en/help/browser/browser.asp}) 
  Cont.}
  \label{tab:the_lines2b}
  \begin{center}
    \begin{tabular}{lll} 
      \hline
      \hline
name &	value &	description \\
      \hline
Mg\_5177    &   5177 &	5176.70 \\
Na\_5896    &   5896 &	5895.60 \\
OI\_6302    &   6302 &	6302.05 \\
OI\_6366    &   6366 &	6365.54 \\
NI\_6529    &   6529 &	6529.03 \\
NII\_6550   &	6550 &	6549.86 \\
Ha\_6565    &   6565 &	6564.61 \\
NII\_6585   &	6585 &	6585.27 \\
Li\_6708    &   6708 &	6707.89 \\
SII\_6718   &	6718 &	6718.29 \\
SII\_6733   &	6733 &	6732.67 \\
CaII\_8500  &   8500 &	8500.36 \\
CaII\_8544  &	8544 &	8544.44 \\
CaII\_8665  &	8665 &	8664.52 \\
      \hline
      \hline
 \end{tabular}
   \end{center}
\end{table}




\noindent
E$+$A (e$+$a) galaxies have... (Roseboom et al. 2006 and refs therein).\\
k$+$a         galaxies have... (Roseboom et al. 2006 and refs therein).\\



\section{References}

\begin{table}
  \caption{The Refs}
  \label{tab:the_ref}
  \begin{center}
    \begin{tabular}{llllll}
      \hline
      \hline 
      Name                            & Year & Journal & Volume & Page & Section(s) \\
      \hline
      Croom et al.                    & 2004 & MNRAS      & 375        & 600  & ???? \\
      Kriek et al.                    & 2007 & astro-ph   & 0611724    & v4   & 3 \\
      Roseboom et al.                 & 2006 & MNRAS      &            &      &   \\
      Baldwin, Phillips \& Terlevich  & 1981& MNRAS      &            &      &   \\
      Yan et al.                      & 2006&            &            &      & \\
      %also K. Brand talk,. Galaxy and BH evolution: Towards a unified view conf
      \hline
      \hline
    \end{tabular}
  \end{center}
\end{table}


\end{document}

